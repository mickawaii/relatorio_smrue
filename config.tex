\documentclass[
	% -- opções da classe memoir --
	12pt,				% tamanho da fonte
	openright,			% capítulos começam em pág ímpar (insere página vazia caso preciso)
	twoside,			% para impressão em verso e anverso. Oposto a oneside
	a4paper,			% tamanho do papel. 
	% -- opções da classe abntex2 --
	%chapter=TITLE,		% títulos de capítulos convertidos em letras maiúsculas
	%section=TITLE,		% títulos de seções convertidos em letras maiúsculas
	%subsection=TITLE,	% títulos de subseções convertidos em letras maiúsculas
	%subsubsection=TITLE,% títulos de subsubseções convertidos em letras maiúsculas
	% -- opções do pacote babel --
	english,			% idioma adicional para hifenização			
	brazil				% o último idioma é o principal do documento
	]{abntex2}

% ---
% Pacotes básicos 
% ---
\usepackage{uarial}
\usepackage{lmodern}
\usepackage{lastpage}			% Usado pela Ficha catalográfica
\usepackage{indentfirst}		% Indenta o primeiro parágrafo de cada seção.
\usepackage[utf8]{inputenc}
\usepackage[T1]{fontenc}
\usepackage[brazilian,hyperpageref]{backref}	 % Paginas com as citações na bibl
\usepackage[alf]{abntex2cite}	% Citações padrão ABNT
\usepackage{microtype} 			% para melhorias de justificação

\usepackage[fixlanguage]{babelbib}
\usepackage{amsmath}
\usepackage[colorinlistoftodos]{todonotes}
\usepackage{tabularx}
\usepackage{multirow}
\usepackage{graphicx}
\usepackage{listings}
\usepackage{url}
\usepackage{float}
\usepackage{comment}
\usepackage{caption}
\usepackage{color}
\usepackage{nomencl}
\usepackage{fontspec}
\usepackage{titlesec}
\definecolor{lightgray}{rgb}{0.95,0.95,0.95}
\linespread{1.25}

% --- 
% CONFIGURAÇÕES DE PACOTES
% --- 

% ---
% Configurações do pacote backref
% Usado sem a opção hyperpageref de backref
%\renewcommand{\backrefpagesname}{Citado na(s) página(s):~}
% Texto padrão antes do número das páginas
%\renewcommand{\backref}{}
% Define os textos da citação
%\renewcommand*{\backrefalt}[4]{
%	\ifcase \#1 %
%		Nenhuma citação no texto.%
%	\or
%		Citado na página \#2.%
%	\else
%		Citado \#1 vezes nas páginas \#2.%
%	\fi}%
% ---

% ---
% Informações de dados para CAPA e FOLHA DE ROSTO
% ---
\titulo{Sistema de Monitoramento Residencial de Uso de Energia}
\autor{Henrique Sussumu Matsui Kano \& Mi Che Li Lee}
\local{São Paulo}
\date{2015}
\orientador{Professor Doutor Carlos Eduardo Cugnasca}
\instituicao{%
  Escola Politécnica da Universidade de São Paulo
  \par
  Departamento de Engenharia da Computação e Sistemas Digitais
  }
\tipotrabalho{Trabalho de formatura}
% O preambulo deve conter o tipo do trabalho, o objetivo, 
% o nome da instituição e a área de concentração 
\preambulo{Trabalho de formatura apresentada à Escola Politécnica da Universidade de São Paulo para a conclusão do curso de graduação em Engenharia de Computação}
% ---

% ---
% Configurações de aparência do PDF final

% alterando o aspecto da cor azul
\definecolor{blue}{RGB}{41,5,195}

% informações do PDF
\makeatletter
\hypersetup{
     	%pagebackref=true,
		pdftitle={\@title}, 
		pdfauthor={\@author},
    	pdfsubject={\imprimirpreambulo},
	    pdfcreator={LaTeX with abnTeX2},
		pdfkeywords={abnt}{latex}{abntex}{abntex2}{trabalho acadêmico}, 
		colorlinks=true,       		% false: boxed links; true: colored links
    	linkcolor=blue,          	% color of internal links
    	citecolor=blue,        		% color of links to bibliography
    	filecolor=magenta,      		% color of file links
		urlcolor=blue,
		bookmarksdepth=4
}
\makeatother
% --- 

% --- 
% Espaçamentos entre linhas e parágrafos 
% --- 

% O tamanho do parágrafo é dado por:
\setlength{\parindent}{1.3cm}

% fonte arial
\setmainfont{Arial}

% Controle do espaçamento entre um parágrafo e outro:
\setlength{\parskip}{0.2cm}  % tente também \onelineskip

% ---
% compila o indice
% ---
\makeindex
% ---


\lstset{%
  backgroundcolor=\color{lightgray},
  commentstyle=\color{gray},
  basicstyle=\footnotesize\ttfamily,
  breaklines=true, 
  tabsize=2,
  captionpos=b
}