\mychapter{Introdução}
\label{Cap:Introducao}

\section{Objetivo}
\label{Sec:objetivo}
\paragraph{
O trabalho propõe a construção de um sistema para monitorar o uso de energia dentro da residência. Para isso optamos por tecnologia de redes sem fio, para a transmissão dos dados coletados dos sensores. Além disso será desenvolvida uma central que reunirá as informações obtidas dos dispositivos distribuídos no ambiente e as apresentará de uma forma simples e intuitiva ao usuário, permitindo o acesso remoto para visualizar o consumo em tempo real do seu domicílio. O sistema dará uma visão quantitativa ao usuário, e isso é fundamental tomada de decisões conscientes, resultando em uma administração eficiente dos gastos residenciais.
}
\section{Motivação}
\label{Sec:motivacao}
\paragraph{
O primeiro dos principais objetivos do grupo foi a preocupação do desenvolvimento de um sistema que englobasse diversas áreas vistas ao longo do curso de engenharia para que fosse possível, ao final do trabalho, uma reflexão pessoal dos integrantes quanto a evolução técnica desses tópicos. O segundo objetivo é a resolução de um problema da atualidade, focada em automação residencial. Tem-se como espectativa um produto resultante aplicável no dia-a-dia e com alguns diferenciais em relação aos produtos existentes no mercado. O projeto propõe um sistema no qual podemos identificar o consumo elétrico por equipamento, ao invés do consumo da rede residencial como um todo; além disso, o software é adaptável às metas de consumo .}
\paragraph{O projeto permite o uso dos conhecimentos vistos em aula relacionados à microeletrônica, redes, sistemas digitais, engenharia de software, banco de dados, entre outros.
}
\section{Justificativa}
\label{Sec:justificativa}
\paragraph{
No contexto atual, a situação energética é destacada pela diminuição da capacidade de produção de energia elétrica devido à escassez de chuvas. Por isso é imprescindível a tomada de atitudes por parte da população para a redução do consumo de eletricidade nas suas residências. Trazendo o problema para a área da engenharia, e sabendo da grande gama de tecnologia disponível, a criação de ferramentas que podem nos auxiliar na monitoração e controle de gastos de energia é favorecida.
}
\paragraph{
Apesar de haver uma grande variedade de sistemas que nos permitem a monitoração do uso de energia, ainda há um fator que desestimula o uso desses na residência, que é o alto custo de instalação devido aos fios dos equipamentos. O projeto proposto utilizará rede sem fio, e para isso, será utilizado o protocolo ZigBee associado aos equipamentos XBee e microcontroladores com o intuito de eliminar esse problema. E além disso o sistema permitirá medir o consumo por aparelho, ao invés do consumo da rede elétrica residencial como um todo, sendo possível, então detectar possíveis aparelhos “vilões”, por excesso de consumo de energia.
}
\paragraph{
Outra necessidade suprida pelo software associado ao sistema é o estabelecimento de metas de consumo por período. O usuário poderá acompanhar o seu consumo graficamente em um período escolhido pelo mesmo e receberá notificações sobre seus gastos e plano de consumo.
}
\section{Organização}
\label{Sec:organizacao}

\todo[inline, color=red!40]{Fazer}