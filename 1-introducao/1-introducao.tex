\mychapter{Introdução}
\label{Cap:Introducao}

\section{Objetivo}
\label{Sec:objetivo}

O trabalho propõe a construção de um sistema para monitorar o uso de energia dentro da residência. O sistema consistirá  de dispositivos interligados por uma rede sem fio, composta por: uma rede local de sensores  que fará o sensoriamento dos parâmetros e se comunicará com uma aplicação em nuvem; e a aplicação em nuvem, que apresentará as métricas enviadas pelos sensores numa interface simples e intuitiva, permitindo também o acesso remoto. 

\section{Motivação}
\label{Sec:motivacao}

A primeira das principais motivações do grupo foi a preocupação do desenvolvimento de um sistema que englobasse diversas áreas vistas ao longo do curso de engenharia para que fosse possível, ao final do trabalho, uma reflexão pessoal dos integrantes quanto a evolução técnica desses tópicos. A segunda motivação é a resolução de um problema muito comum na atualidade, que é o encarecimento da energia,  utilizando como instrumentos as técnicas de automação residencial.

Tem-se como expectativa um produto facilmente aplicável no dia-a-dia e com alguns diferenciais em relação aos produtos existentes no mercado \cite{green_ant_site} \cite{neurio_site} \cite{emonpi_site} \cite{ecoisme_site}, no sentido que usará equipamentos relativamente fáceis de obter com um resultado satisfatório. O projeto propõe um sistema no qual podemos identificar o consumo elétrico por equipamento, ao invés do consumo da rede residencial como um todo e, além disso, ser adaptável às metas de consumo, considerando a experiência do usuário.

\section{Justificativa}
\label{Sec:justificativa}

Em 2015, devido à escassez de chuvas, houve uma queda significativa no nível dos reservatórios das principais hidrelétricas do Brasil e o uso mais intenso de termelétricas. Isso provocou reajustes altos, encarecendo a energia do país e o custo foi repassado para os consumidores finais \cite{news_g1, news_secretaria_de_energia}. Por isso é imprescindível a tomada de atitudes por parte da população tanto para controlar os gastos na conta de luz quanto para a redução do consumo de eletricidade nas suas residências,  aliviando a carga do sistema de produção e distribuição de eletricidade.

Trazendo o problema para a área da engenharia, e sabendo da grande gama de tecnologia disponível, a criação de ferramentas que podem nos auxiliar na monitoração e controle de gastos de energia é favorecida. Existem sistemas no mercado, ou prestes a entrar no mercado, que realiza a função de monitorar o consumo de energia residencialmente como a OpenEnergyMonitor\cite{open_energy_monitor}, Neurio\cite{neurio_site}, Green Ant\cite{green_ant_site}. Entretanto há a preocupação de se desenvolver um sistema, em alguns meses, aproveitando a onda de desenvolvimento e hardware/software open-source, com padronizações e a comercialização de tecnologias de redes de sensores sem fio. Um sistema pode ser assim construído modularmente, permitindo o desenvolvimento rápido de protótipos altamente personalizáveis, de pouco custo e que consomem pouca bateria.

O sistema permitirá medir o consumo por aparelho, ao invés do consumo da rede elétrica residencial como um todo, sendo possível, então detectar possíveis aparelhos “vilões”, por excesso de consumo de energia.

O sistema dará uma visão quantitativa ao usuário, e isso é fundamental tomada de decisões conscientes, resultando em uma administração eficiente dos gastos residenciais.

\section{Organização}
\label{Sec:organizacao}

O documento segue o seguinte formato: no capítulo um são apresentados os conceitos do projeto e diagramas do sistema de uma forma genérica, sem mencionar nomes ou marcas de componentes, porém são mencionando os principais módulos do sistema, assim como os nomes utilizados para esses módulos no trabalho todo.

No capítulo dois são listadas e apresentadas as peças e softwares que compõem o sistema própriamente ditos. 

No capítulo quatro é apresentado o método de projeto adotada pelo grupo durante o desenvolvimento do projeto, da parte de projeto até sua implementação e conclusão.

No capítulo cinco é detalhado mais sobre como foi feita a implementação, citando detalhes mais técnicos de problemas, soluções e mudanças de projeto.

No capítulo seis é detalhado o procedimento de aceitação do sistema.

No capítulo sete são apontadas algumas considerações finais como comentários e resultados atingidos.

Além do capítulo sete serão colocados anexos extras que são considerados úteis para o entendimento do trabalho.