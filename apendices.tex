\apendices
\chapter{Orçamento das peças}
\begin{table}
\centering
{\renewcommand{\arraystretch}{1.5}
\renewcommand{\tabcolsep}{0.2cm}
\begin{tabular}{|l|r|r|r|r|r|}
\hline
Componentes & qtd & Total(R\$) & Total(US\$) \\
\hline
XBee 2mW PCB Antenna & 4 & {} & US\$103,80 \\
XBee Explorer Dongle & 1 & {} & US\$24,95 \\
Kit Raspberry Pi2 & 1 & R\$309,89 & {} \\
Sensores de Corrente & 3 & {} & US\$29,85 \\
XBee Shield & 3 & {} & US\$44,85 \\
Arduino Stackable Header Kit & 4 & {} & US\$6,00 \\
Arduino Uno - R3 & 3 & R\$ 137,70 & {} \\
\hline
\multicolumn{2}{|l|}{Totais} & R\$447,59 & US\$209,45 \\
\hline
\end{tabular}}
\caption{\label{tab:orcamento} Orçamento das peças necessárias.}
\end{table}

\chapter{Descrição dos casos de uso}
\label{apendice_caso_uso}
\subsection{Casos de uso}

Os casos de uso do sistema estão listados na tabela \ref{tab:casos_de_uso}.

\begin{table}
\centering
{\renewcommand{\arraystretch}{1.5}
\renewcommand{\tabcolsep}{0.2cm}
\begin{tabular}{|c|c|}
\hline
\textbf{Funções} & \textbf{Casos de uso} \\
\hline
\multirow{4}{*}{Gerenciar conta} & Fazer cadastro\\
& Fazer login\\
& Fazer logout\\
& Recuperar senha\\
\hline
\multirow{3}{*}{Gerenciar equipamentos} & Criar equipamento\\
& Editar equipamento\\
& Remover equipamento\\
\hline
\multirow{3}{*}{Gerenciar sensores} & Detectar sensores\\
& Editar sensor\\
& Remover sensor\\
\hline
\multirow{3}{*}{Gerenciar metas} & Criar meta\\
& Editar meta\\
& Remover meta\\
\hline
\multirow{4}{*}{Gerenciar consumo} & Criar consumo\\
& Visualizar consumo\\
& Importar consumo\\
& Exportar consumo\\
\hline
Atualizar taxas da AES & Atualizar taxas da AES\\
\hline
Configurar sistema & Configurar sistema\\
\hline
\end{tabular}}
\caption{\label{tab:casos_de_uso} Casos de Uso.}
\end{table}
%
\subsection{Descrição dos casos de uso}

A seguir são descritos os casos de uso do sistema. 

% ************************************************
% 1 - GERENCIAR CONTA
% ************************************************
\subsubsection{Caso de Uso 1: Gerenciar conta}
\subsubsection{Caso de Uso 1.1: Fazer cadastro}
\begin{description}
	\item[Descrição:] inserção de um novo usuário comum no sistema
	\item[Evento iniciador:] solicitação de cadastro
	\item[Atores:] usuário
	\item[Pré-condição:] sistema exibindo tela de solicitação de cadastro
	\item[Sequência de eventos:] \hfill
		\begin{enumerate}
			\item{Usuário solicita cadastro}
			\item{Sistema exibe o formulário de cadastro}
			\item{Usuário insere os seus dados}
			\item{Sistema insere o novo usuário e exibe o resultado}
		\end{enumerate}
	\item[Pós-condição:] novo usuário cadastrado, usuário é logado automaticamente e é exibida a tela inicial
	\item[Extensões:] \hfill
		\begin{enumerate}
			\item{\textbf{Usuário a ser cadastrado já existe:} sistema apresenta uma mensagem ao usuário (passo 4)}
			\item{\textbf{Dados do usuário não consistentes:} sistema apresenta mensagem de erro ao usuário (passo 4)}
		\end{enumerate}
	\item[Inclusões:] \hfill
		\begin{enumerate}
			\item{Buscar usuário (passo 4)}
		\end{enumerate}
\end{description}
%
\subsubsection{Caso de Uso 1.2: Fazer login}
\begin{description}
	\item[Descrição:] criar uma sessão do usuário no sistema
	\item[Evento iniciador:] solicitação de login
	\item[Atores:] usuário
	\item[Pré-condição:] usuário cadastrado e não há usuário logado
	\item[Sequência de eventos:] \hfill
		\begin{enumerate}
			\item{usuário solicita login}
			\item{sistema exibe formulário para login}
			\item{usuário insere os dados de login}
			\item{sistema cria uma sessão para o usuário e redireciona para a página inicial}
		\end{enumerate}
	\item[Pós-condição:] sessão criada e sistema exibe a tela inicial
	\item[Extensões:] \hfill
		\begin{enumerate}
			\item{\textbf{Usuário não encontrado:} sistema apresenta uma mensagem de erro ao usuário (passo 4)}
			\item{\textbf{Dados não consistentes:} sistema apresenta uma mensagem de erro ao usuário (passo 4)}
		\end{enumerate}
	\item[Inclusões:] \hfill
		\begin{enumerate}
			\item{Buscar usuário (passo 4)}
		\end{enumerate}
\end{description}
%
\subsubsection{Caso de Uso 1.3: Fazer logout}
\begin{description}
	\item[Descrição:] encerrar a sessão do usuário atual no sistema
	\item[Evento iniciador:] solicitação de logout
	\item[Atores:] usuário
	\item[Pré-condição:] usuário logado
	\item[Sequência de eventos:] \hfill
		\begin{enumerate}
			\item{usuário solicita logout}
			\item{sistema encerra a sessão atual, e redireciona para a página de login}
		\end{enumerate}
	\item[Pós-condição:] sessão encerrada e sistema exibe tela de login
\end{description}
%
\subsubsection{Caso de Uso 1.4: Recuperar senha}
\begin{description}
	\item[Descrição:] recuperar a senha do usuário
	\item[Evento iniciador:] solicitação de recuperação de senha
	\item[Atores:] usuário
	\item[Pré-condição:] usuário cadastrado, não há usuário logado e sistema exibindo tela de login
	\item[Sequência de eventos:] \hfill
		\begin{enumerate}
			\item{usuário solicita recuperação de senha}
			\item{sistema exibe formulário para recuperação de senha}
			\item{usuário insere o e-mail}
			\item{sistema envia e-mail para recuperar a senha e exibe mensagem}
			\item{usuário clica no link para recuperar senha no e-mail}
			\item{sistema exibe o formulário para recuperar a senha}
			\item{usuário insere os dados pedidos}
			\item{sistema atualiza a senha do usuário, autentica o usuário e redireciona para a tela inicial}
		\end{enumerate}
	\item[Pós-condição:] senha do usuário atualizada, usuário autenticado e sistema mostra a tela inicial
	\item[Extensões:] \hfill
		\begin{enumerate}
			\item{\textbf{Dados não consistentes:} sistema apresenta uma mensagem de erro ao usuário (passo 4, 8)}
			\item{\textbf{Senha antiga incorreta:} sistema apresenta uma mensagem de erro ao usuário (passo 8)}
		\end{enumerate}
	\item[Inclusões:] \hfill
		\begin{enumerate}
			\item{Buscar usuário (passo 8)}
		\end{enumerate}
\end{description}
% ************************************************
% 2 - GERENCIAR EQUIPAMENTO
% ************************************************
\subsubsection{Caso de Uso 2: Gerenciar equipamentos}
\subsubsection{Caso de Uso 2.1: Criar equipamento}
\begin{description}
	\item[Descrição:] criar um novo equipamento
	\item[Evento iniciador:] solicitação de criação de equipamento
	\item[Atores:] usuário
	\item[Pré-condição:] usuário logado e sistema exibindo listagem de equipamentos
	\item[Sequência de eventos:] \hfill
		\begin{enumerate}
			\item{usuário solicita criação de equipamento}
			\item{sistema exibe formulário para criação}
			\item{usuário insere os dados para criação}
			\item{sistema cria um equipamento e redireciona para a listagem de equipamentos}
		\end{enumerate}
	\item[Pós-condição:] equipamento criado e sistema exibe listagem de equipamentos
	\item[Extensões:] \hfill
		\begin{enumerate}
			\item{\textbf{Dados não consistentes:} sistema apresenta uma mensagem de erro ao usuário (passo 4)}
			\item{\textbf{Equipamento já existe:} sistema apresenta uma mensagem de erro ao usuário (passo 4)}
		\end{enumerate}
	\item[Inclusões:] \hfill
		\begin{enumerate}
			\item{Buscar equipamento (passo 4)}
		\end{enumerate}
\end{description}
%
\subsubsection{Caso de Uso 2.2: Editar equipamento}
\begin{description}
	\item[Descrição:] editar um equipamento
	\item[Evento iniciador:] solicitação de edição de equipamento
	\item[Atores:] usuário
	\item[Pré-condição:] usuário logado, existem equipamentos e sistema exibindo listagem de equipamentos
	\item[Sequência de eventos:] \hfill
		\begin{enumerate}
			\item{usuário seleciona o equipamento desejado para edição}
			\item{sistema exibe formulário para edição}
			\item{usuário altera os dados desejados}
			\item{sistema atualiza o equipamento e redireciona para a listagem de equipamentos}
		\end{enumerate}
	\item[Pós-condição:] equipamento atualizado e sistema exibe listagem de equipamentos
	\item[Extensões:] \hfill
		\begin{enumerate}
			\item{\textbf{Dados não consistentes:} sistema apresenta uma mensagem de erro ao usuário (passo 4)}
		\end{enumerate}
	\item[Inclusões:] \hfill
		\begin{enumerate}
			\item{Buscar equipamento (passo 2, 4)}
		\end{enumerate}
\end{description}
%
\subsubsection{Caso de Uso 2.3: Remover equipamento}
\begin{description}
	\item[Descrição:] remover um equipamento
	\item[Evento iniciador:] solicitação de remoção de equipamento
	\item[Atores:] usuário
	\item[Pré-condição:] usuário logado, existem equipamentos e sistema exibindo listagem de equipamentos
	\item[Sequência de eventos:] \hfill
		\begin{enumerate}
			\item{usuário seleciona o equipamento desejado para remoção}
			\item{sistema pede confirmação para remoção}
			\item{usuário confirma}
			\item{sistema remove o equipamento e redireciona para a listagem de equipamentos}
		\end{enumerate}
	\item[Pós-condição:] equipamento removido e sistema exibe listagem de equipamentos
	\item[Extensões:] \hfill
		\begin{enumerate}
			\item{\textbf{Usuário não confirma:} sistema não remove e volta para a tela de listagem (passo 4)}
		\end{enumerate}
	\item[Inclusões:] \hfill
		\begin{enumerate}
			\item{Buscar equipamento (passo 2, 4)}
		\end{enumerate}
\end{description}
% ************************************************
% 3 - GERENCIAR SENSORES
% ************************************************
\subsubsection{Caso de Uso 3: Gerenciar sensores}
\subsubsection{Caso de Uso 3.1: Detectar sensor}
\begin{description}
	\item[Descrição:] detectar um sensor
	\item[Evento iniciador:] solicitação de detecção de sensores
	\item[Atores:] usuário
	\item[Pré-condição:] usuário logado e sistema exibindo listagem de sensores
	\item[Sequência de eventos:] \hfill
		\begin{enumerate}
			\item{usuário solicita detecção de sensor}
			\item{sistema detecta e cria um sensor no sistema com status ativo e atualiza a lista de sensores}
		\end{enumerate}
	\item[Pós-condição:] sensor criado e sistema exibe listagem de sensores
	\item[Extensões:] \hfill
		\begin{enumerate}
			\item{\textbf{Sensor já existe no sistema:} sistema atualiza o status do sensor para ativo (passo 2)}
		\end{enumerate}
	\item[Inclusões:] \hfill
		\begin{enumerate}
			\item{Buscar sensor (passo 2)}
		\end{enumerate}
\end{description}
%
\subsubsection{Caso de Uso 3.2: Editar sensor}
\begin{description}
	\item[Descrição:] editar um sensor
	\item[Evento iniciador:] solicitação de edição de sensor
	\item[Atores:] usuário
	\item[Pré-condição:] usuário logado, existem sensores e sistema exibindo listagem de sensores
	\item[Sequência de eventos:] \hfill
		\begin{enumerate}
			\item{usuário seleciona o sensor desejado para edição}
			\item{sistema exibe formulário para edição}
			\item{usuário altera os dados desejados}
			\item{sistema atualiza o sensor e redireciona para a listagem de sensores}
		\end{enumerate}
	\item[Pós-condição:] sensor atualizado e sistema exibe listagem de sensores
	\item[Extensões:] \hfill
		\begin{enumerate}
			\item{\textbf{Dados não consistentes:} sistema apresenta uma mensagem de erro ao usuário (passo 4)}
		\end{enumerate}
	\item[Inclusões:] \hfill
		\begin{enumerate}
			\item{Buscar sensor (passo 2, 4)}
		\end{enumerate}
\end{description}
%
\subsubsection{Caso de Uso 3.3: Remover sensor}
\begin{description}
	\item[Descrição:] remover um sensor
	\item[Evento iniciador:] solicitação de remoção de sensor
	\item[Atores:] usuário
	\item[Pré-condição:] usuário logado, existem sensores e sistema exibindo listagem de sensores
	\item[Sequência de eventos:] \hfill
		\begin{enumerate}
			\item{usuário seleciona o sensor desejado para remoção}
			\item{sistema pede confirmação para remoção}
			\item{usuário confirma}
			\item{sistema remove o sensor e redireciona para a listagem de sensores}
		\end{enumerate}
	\item[Pós-condição:] sensor removido e sistema exibe listagem de sensores
	\item[Extensões:] \hfill
		\begin{enumerate}
			\item{\textbf{Usuário não confirma:} sistema não remove e volta para a tela de listagem (passo 4)}
		\end{enumerate}
	\item[Inclusões:] \hfill
		\begin{enumerate}
			\item{Buscar sensor (passo 2, 4)}
		\end{enumerate}
\end{description}
% ************************************************
% 4 - GERENCIAR META
% ************************************************
\subsubsection{Caso de Uso 4: Gerenciar metas}
\subsubsection{Caso de Uso 4.1: Criar meta}
\begin{description}
	\item[Descrição:] criar uma nova meta
	\item[Evento iniciador:] solicitação de criação de meta
	\item[Atores:] usuário
	\item[Pré-condição:] usuário logado e sistema exibindo listagem de metas
	\item[Sequência de eventos:] \hfill
		\begin{enumerate}
			\item{usuário solicita criação de meta}
			\item{sistema exibe formulário para criação}
			\item{usuário insere os dados para criação}
			\item{sistema cria uma meta e redireciona para a listagem de metas}
		\end{enumerate}
	\item[Pós-condição:] meta criada e sistema exibe listagem de metas
	\item[Extensões:] \hfill
		\begin{enumerate}
			\item{\textbf{Dados não consistentes:} sistema apresenta uma mensagem de erro ao usuário (passo 4)}
			\item{\textbf{Meta já existe:} sistema apresenta uma mensagem de erro ao usuário (passo 4)}
		\end{enumerate}
	\item[Inclusões:] \hfill
		\begin{enumerate}
			\item{Buscar meta (passo 4)}
		\end{enumerate}
\end{description}
%
\subsubsection{Caso de Uso 4.2: Editar meta}
\begin{description}
	\item[Descrição:] editar uma meta
	\item[Evento iniciador:] solicitação de edição de meta
	\item[Atores:] usuário
	\item[Pré-condição:] usuário logado, existem metas e sistema exibindo listagem de metas
	\item[Sequência de eventos:] \hfill
		\begin{enumerate}
			\item{usuário seleciona a meta desejado para edição}
			\item{sistema exibe formulário para edição}
			\item{usuário altera os dados desejados}
			\item{sistema atualiza a meta e redireciona para a listagem de metas}
		\end{enumerate}
	\item[Pós-condição:] meta atualizada e sistema exibe listagem de metas
	\item[Extensões:] \hfill
		\begin{enumerate}
			\item{\textbf{Dados não consistentes:} sistema apresenta uma mensagem de erro ao usuário (passo 4)}
		\end{enumerate}
	\item[Inclusões:] \hfill
		\begin{enumerate}
			\item{Buscar meta (passo 2, 4)}
		\end{enumerate}
\end{description}
%
\subsubsection{Caso de Uso 4.3: Remover meta}
\begin{description}
	\item[Descrição:] remover uma meta
	\item[Evento iniciador:] solicitação de remoção de meta
	\item[Atores:] usuário
	\item[Pré-condição:] usuário logado, existem metas e sistema exibindo listagem de metas
	\item[Sequência de eventos:] \hfill
		\begin{enumerate}
			\item{usuário seleciona a meta desejado para remoção}
			\item{sistema pede confirmação para remoção}
			\item{usuário confirma}
			\item{sistema remove a meta e redireciona para a listagem de metas}
		\end{enumerate}
	\item[Pós-condição:] meta removida e sistema exibe listagem de metas
	\item[Extensões:] \hfill
		\begin{enumerate}
			\item{\textbf{Usuário não confirma:} sistema não remove e volta para a tela de listagem (passo 4)}
		\end{enumerate}
	\item[Inclusões:] \hfill
		\begin{enumerate}
			\item{Buscar meta (passo 2, 4)}
		\end{enumerate}
\end{description}
% ************************************************
% 5 - GERENCIAR CONSUMO
% ************************************************
\subsubsection{Caso de Uso 5: Gerenciar consumos}
\subsubsection{Caso de Uso 5.1: Criar consumo}
\begin{description}
	\item[Descrição:] inserir consumos no sistema
	\item[Evento iniciador:] solicitação para criação de consumo
	\item[Atores:] módulo coordenador
	\item[Pré-condição:] módulo coordenador ligado e sistema online
	\item[Sequência de eventos:] \hfill
		\begin{enumerate}
			\item{módulo coordenador solicita criação de consumo}
			\item{sistema cria o consumo}
		\end{enumerate}
	\item[Pós-condição:] consumo criado
	\item[Extensões:] \hfill
		\begin{enumerate}
			\item{\textbf{Perda de conexão:} consumo não é criado (passo 2)}
		\end{enumerate}
\end{description}
%
\subsubsection{Caso de Uso 5.2: Visualizar consumo}
\begin{description}
	\item[Descrição:] visualizar os consumos na forma de gráficos
	\item[Evento iniciador:] solicitação de geração de gráfico
	\item[Atores:] usuário
	\item[Pré-condição:] usuário logado, existem consumos e sistema exibindo tela de consumo
	\item[Sequência de eventos:] \hfill
		\begin{enumerate}
			\item{usuário configura os parâmetros e solicita geração do gráfico}
			\item{sistema exibe o gráfico de consumo}
		\end{enumerate}
	\item[Pós-condição:] sistema exibe gráfico de consumo
	\item[Extensões:] \hfill
		\begin{enumerate}
			\item{\textbf{Dados não consistentes:} sistema apresenta uma mensagem de erro ao usuário (passo 2)}
		\end{enumerate}
	\item[Inclusões:] \hfill
		\begin{enumerate}
			\item{Buscar consumos (passo 2)}
		\end{enumerate}
\end{description}
%
\subsubsection{Caso de Uso 5.3: Importar consumos}
\begin{description}
	\item[Descrição:] importar consumos por csv
	\item[Evento iniciador:] solicitação de importação de consumos
	\item[Atores:] usuário
	\item[Pré-condição:] usuário logado, sistema exibindo tela de consumo
	\item[Sequência de eventos:] \hfill
		\begin{enumerate}
			\item{usuário insere o arquivo csv e solicita importação de consumos}
			\item{sistema lê o csv, cria os consumos e exibe tela de consumo}
		\end{enumerate}
	\item[Pós-condição:] novos consumos criados e sistema exibe tela de consumo
	\item[Extensões:] \hfill
		\begin{enumerate}
			\item{\textbf{Dados não consistentes:} sistema apresenta uma mensagem de erro ao usuário (passo 2)}
		\end{enumerate}
\end{description}
%
\subsubsection{Caso de Uso 5.4: Exportar consumos}
\begin{description}
	\item[Descrição:] exportar consumos por csv
	\item[Evento iniciador:] solicitação de exportação de consumos
	\item[Atores:] usuário
	\item[Pré-condição:] usuário logado, existem consumos e sistema exibindo tela de consumo
	\item[Sequência de eventos:] \hfill
		\begin{enumerate}
			\item{usuário seleciona o período desejado do consumo para exportação}
			\item{sistema disponibiliza o download do csv}
		\end{enumerate}
	\item[Pós-condição:] sistema exibe arquivo de csv
	\item[Inclusões:] \hfill
		\begin{enumerate}
			\item{Buscar consumo (passo 2)}
		\end{enumerate}
\end{description}
% ************************************************
% 6 - ATUALIZAR TAXAS DA AES
% ************************************************
\subsubsection{Caso de Uso 6: Atualizar taxas da AES}
\begin{description}
	\item[Descrição:] atualizar dados de custo da AES Eletropaulo no sistema
	\item[Evento iniciador:] solicitação de atualização das taxas
	\item[Atores:] usuário
	\item[Pré-condição:] usuário logado e sistema exibindo tela de listagem de taxas
	\item[Sequência de eventos:] \hfill
		\begin{enumerate}
			\item{usuário solicita atualização de taxas}
			\item{sistema busca dados do site da AES Eletropaulo e cria taxas no sistema}
		\end{enumerate}
	\item[Pós-condição:] taxas criadas e sistema exibe listagem de taxas
	\item[Extensões:] \hfill
		\begin{enumerate}
			\item{\textbf{Taxa já existe no sistema:} sistema atualiza a taxa correspondente (passo 2)}
		\end{enumerate}
	\item[Inclusões:] \hfill
		\begin{enumerate}
			\item{Buscar taxa (passo 2)}
		\end{enumerate}
\end{description}
% ************************************************
% 7 - CONFIGURAR SISTEMA
% ************************************************
\subsubsection{Caso de Uso 7: Configurar sistema}
\begin{description}
	\item[Descrição:] mudar configuração do sistema
	\item[Evento iniciador:] solicitação de mudança de configuração do sistema
	\item[Atores:] usuário
	\item[Pré-condição:] usuário logado e sistema exibindo tela de configuração
	\item[Sequência de eventos:] \hfill
		\begin{enumerate}
			\item{usuário muda a configuração e solicita salvar a configuração}
			\item{sistema salva as configurações e retorna para tela de configuração}
		\end{enumerate}
	\item[Pós-condição:] configurações salvas e sistema mostra tela de configuração
	\item[Extensões:] \hfill
		\begin{enumerate}
			\item{\textbf{Dados inconsistentes:} sistema mostra mensagem de erro ao usuário (passo 2)}
		\end{enumerate}
	\item[Inclusões:] \hfill
		\begin{enumerate}
			\item{Buscar configuração do usuário (passo 2)}
		\end{enumerate}
\end{description}

\chapter{Descrição das classes}
\label{apendice_classes}
\section{Equipment}
\begin{description}
  \item[Classe:] Equipment
  \item[Descrição:] Representa um equipamento na aplicação. Os equipamentos são criados pelos usuários dentro do sistema. É necessário possuir um sensor associado para que o equipamento possa ser associado a um consumo.
  \item[Atributos:] \hfill
    \begin{enumerate}
      \item id (integer): identificador do equipamento
      \item name (String): nome do equipamento
      \item description (Text): descrição do equipamento 
      \item nominal\_power (float): potência nominal do equipamento 
      \item measurement\_unit (String): unidade de medida utilizada em nominal\_power
      \item approximated\_consumption (float): consumo aproximado do equipamento dado pelo fabricante 
    \end{enumerate}
  \item[Relacionamentos:] \hfill
    \begin{enumerate}
      \item um equipamento possui nenhum ou um sensor
      \item um equipamento possui nenhum ou vários consumos
      \item um equipamento possui nenhum ou várias metas
    \end{enumerate}
\end{description} 
%
\section{Sensor}
\begin{description}
  \item[Classe:] Sensor
  \item[Descrição:] Representa um sensor na aplicação. Os sensores são criados automaticamente pelo sistema ao receber um consumo de um sensor não registrado. O usuário poderá, então, editar o nome do sensor. Porém, como não há dado que indique em qual aparelho o sensor foi instalado, tal associação deve ser feita através de configuração (Caso de uso Configurar sistema). Caso um sensor seja alocado de um equipamento para outro, os novos consumos passarão a pertencer ao segundo equipamento.
  \item[Atributos:] \hfill
    \begin{enumerate}
      \item id (integer): identificador do sensor
      \item name (String): nome dado pelo usuário para o sensor
      \item code (String): identificador do sensor, enviada pelo módulo sensor (endereço MAC do XBee no módulo sensor)
      \item equipment\_id (integer): equipamento ao qual está associado
    \end{enumerate}
  \item[Relacionamentos:] \hfill
    \begin{enumerate}
      \item um sensor pertence a um ou nenhum equipamento
    \end{enumerate}
\end{description} 
%
\section{Consumption}
\begin{description}
  \item[Classe:] Consumption
  \item[Descrição:] Representa uma medida de consumo feita de um equipamento em um dado instante. Quando um consumo é enviado ao sistema, o valor da corrente, tensão e identificador do sensor são enviados. Caso o identificador do sensor não exista dentro do sistema, uma nova instância de sensor será criada. A partir do momento em que o sensor tiver um equipamento associado, consumos para aquele equipamento poderão ser criados. Caso um sensor seja alocado de um equipamento para outro, os consumos para o equipamento anterior vão continuar pertencendo ao mesmo, enquanto os novos consumos pertencerão ao segundo equipamento.
  \item[Atributos:] \hfill
    \begin{enumerate}
      \item id (integer): identificador do consumo
      \item equipment\_id (integer): identificador do equipamento
      \item moment (DateTime): a data e a hora de quando foi feita a medida
      \item current (float): corrente no momento da medida em amperes
      \item voltage (float): tensão da tomada do equipamento. 220V ou 127V
    \end{enumerate}
  \item[Relacionamentos:] \hfill
    \begin{enumerate}
      \item um consumo pertence a um equipamento
    \end{enumerate}
\end{description} 
%
\section{User}
\begin{description}
  \item[Classe:] User
  \item[Descrição:] Representa um usuário do sistema. 
  \item[Atributos:] \hfill
    \begin{enumerate}
      \item id (integer): identificador do usuário
      \item name (String):  nome do usuário
      \item username (String): nome de usuário usado para efetuar o login
      \item password (String encriptado): senha do usuário usada para efetuar o login
        \item income\_type (String): o tipo de renda do usuário, Residencial ou Residencial de baixa renda, de acordo com a especificação da AES eletropaulo.
    \end{enumerate}
\end{description} 
%
\section{Goal}
\begin{description}
  \item[Classe:] Goal
  \item[Descrição:] Representa uma meta de consumo para um mês. Ao cadastrar a meta, ela calcula um valor igual a percentagem (value\_in\_percent) do total de consumo para um equipamento no mês anterior. Ao ser traçado o gráfico do mês pertencente ao da meta para aquele equipamento, um gráfico com o valor da meta será exibido.
  \item[Atributos:] \hfill
    \begin{enumerate}
      \item id (integer): identificador da meta
      \item equipment\_id (integer): identificador do equipamento
      \item name (String):  nome da meta
      \item value\_in\_percent (float): consumo pretendido em percentagem (em relação ao mês anterior)
      \item value\_absolute (float): consumo pretendido (em relação ao mês anterior)
      \item yearmonth\_start (DateTime): início do período da meta
        \item yearmonth\_end (DateTime): fim do período da meta
    \end{enumerate}
  \item[Relacionamentos:] \hfill
    \begin{enumerate}
      \item uma meta pertence a um equipamento
    \end{enumerate}
\end{description} 
%
\section{AESRate}
\begin{description}
  \item[Classe:] AESRate
  \item[Descrição:] Representa a taxa de conversão da AES eletropaulo de kilowatts hora para reais. Esses valores são obtidos através da página de tarifas do site da AES Eletropaulo \cite{aes_site}. Caso o usuário queira visualizar o consumo em reais, o usuário deve escolher a opção de integrar o gráfico também (pois as tarifas são calculadas em função da energia consumida, e não em função potência consumida em dado instante). Em seguida, o sistema identifica qual taxa de conversão, dependendo da data do consumo (deve ser maior ou igual a valid\_date), tipo de renda (se é igual ao atributo name), faixa de consumo (o valor de consumo deve ser maior ou igual a range\_start e menor ou igual a range\_end). Identificada a taxa, o sistema multiplica o valor de cada ponto do gráfico com a soma de TE e TUSD da taxa para converter em reais.
  \item[Atributos:] \hfill
    \begin{enumerate}
      \item id (integer): identificador da taxa de conversão
      \item name (String): nome da taxa de conversão, o mesmo utilizado pela AES.
      \item te (float): tarifa de energia
      \item tusd (float): tarifa de uso do sistema de distribuição
      \item date (DateTime): o instante que a taxa de conversão foi buscada
      \item valid\_date (DateTime): data de início da validade das tarifas
      \item range\_start (float): o início da faixa de consumo que define a taxa de conversão
        \item range\_end (float): o fim da faixa de consumo que define a taxa de conversão
    \end{enumerate}
\end{description} 

\chapter{Testes do software}
\label{apendice_software_tests}
\begin{description}
  \section{Caso de teste 1: Realizar Login}
  \item[Testes feitos:]
  \begin{enumerate}
    \item{fazer login com nome de usuário errado}
    \item{fazer login com senha errada}
    \item{fazer login com informações certas}
  \end{enumerate}
  \item[Problemas:]
  \begin{enumerate}
    \item{Nos testes 3 e 4 não aparecem mensagens de erro}
  \end{enumerate}
  \item[Correções feitas:]
  \begin{enumerate}
    \item{Colocada uma mensagem de erro quando uma identificação está errada}
  \end{enumerate}

  \section{Caso de teste 2: importar os dados de consumo de um equipamento}
  \item[Testes feitos:]
  \begin{enumerate}
    \item{Importação de um arquivo qualquer}
    \item{Importação de um csv com formatação errada no data}
    \item{Importação de um csv usando virgulas e pontos como separadores decimais}
    \item{Importação de um csv formatado corretamente}
  \end{enumerate}
  \item[Problemas:]
  \begin{enumerate}
    \item{Nos testes 2 e 3 não aparecem mensagens de erro legíveis}
    \item{Nos testes 4, existem problemas de horários inexistentes devido ao horário de verão}
  \end{enumerate}
  \item[Correções feitas:]
  \begin{enumerate}
    \item{Adição de menssagens de erro da linha do arquivo e qual o erro gerado}
  \end{enumerate}

  \section{Caso de teste 3: Visualizar o consumo através de gráficos por equipamento ou total}
  \item[Testes feitos:]
  Com o banco populado com medidas:
  \begin{enumerate}
    \item{Visualização de dados por hora}
    \item{Visualização de dados por dia}
    \item{Visualização de dados por mês}
    \item{Visualização de metas de consumo}
    \item{Visualização de dados por hora integrado}
    \item{Visualização de dados por hora em reais}
    \item{Visualização de dados por hora de três equipamentos diferentes}
  \end{enumerate}
  \item[Problemas:]
  \begin{enumerate}
    \item{Nos teste 1, devido a como foi implementado, o gráfico não mostra os dados por hora, mas pela granularidade de tempo em que os dados se encontram no banco de dados}
  \end{enumerate}
  \item[Correções feitas:]
  \begin{enumerate}
    \item{Corrigido a maneira que a API calcula os dados por hora para levar em consideração o tempo e não apenas devolver os dados que se encontram entre os limites dados}
  \end{enumerate}

  \section{Caso de teste 4: Obtenção dos dados da AES eletropaulo}
  \item[Testes feitos:]
  \begin{enumerate}
    \item{Obtenção de dados em um dia qualquer}
    \item{Obtenção de dados após a data de validade do primeiro teste}
  \end{enumerate}
  \item[Problemas:]
  \begin{enumerate}
    \item{No teste 2, foi descoberto que como a implementação é dependente da estrutura do site da eletropaulo, caso essa estrutura mude, a obtenção de dados não funciona}
  \end{enumerate}
  \item[Correções feitas:]
  \begin{enumerate}
    \item{O problema só é solúvel com manutenções periódicas}
  \end{enumerate}


  \section{Caso de teste 5: Criar equipamento}
  \item[Testes feitos:]
  \begin{enumerate}
    \item{Criação da entidade com um nominal\_power (float) com letras}
    \item{Criação da entidade com um approximated\_consumption (float) com letras}
    \item{Criação da entidade com informações válidas}
  \end{enumerate}
  \item[Problemas:]
  \begin{enumerate}
    \item{Nos testes 1 e 2, não criou mas também não avisou quais foram os erros}
  \end{enumerate}
  \item[Correções feitas:]
  \begin{enumerate}
    \item{Adição das mensagens de erro para os dois casos de teste 1 e 2}
  \end{enumerate}

  \section{Caso de teste 6: Modificar equipamento}
  \item[Testes feitos:]
  \begin{enumerate}
    \item{Modificação da entidade com um nominal\_power (float) com letras}
    \item{Modificação da entidade com um approximated\_consumption (float) com letras}
  \end{enumerate}
  \item[Problemas:]
  \begin{enumerate}
    \item{-}
  \end{enumerate}
  \item[Correções feitas:]
  \begin{enumerate}
    \item{-}
  \end{enumerate}  

  \section{Caso de teste 7: Excluir equipamento}
  \item[Testes feitos:]
  \begin{enumerate}
    \item{Exclusão de uma entidade}
  \end{enumerate}
  \item[Problemas:]
  \begin{enumerate}
    \item{-}
  \end{enumerate}
  \item[Correções feitas:]
  \begin{enumerate}
    \item{-}
  \end{enumerate}

  \section{Caso de teste 8: Listagem de equipamentos}
  \item[Testes feitos:]
  \begin{enumerate}
    \item{Visualização de uma entidade}
  \end{enumerate}
  \item[Problemas:]
  \begin{enumerate}
    \item{-}
  \end{enumerate}
  \item[Correções feitas:]
  \begin{enumerate}
    \item{-}
  \end{enumerate}

  \section{Caso de teste 9: Criar sensor}
  \item[Testes feitos:]
  \begin{enumerate}
    \item{Introdução de dois módulos sensores novos no sistema para cadastro automático}
  \end{enumerate}
  \item[Problemas:]
  \begin{enumerate}
    \item{-}
  \end{enumerate}
  \item[Correções feitas:]
  \begin{enumerate}
    \item{-}
  \end{enumerate}  

  \section{Caso de teste 10: Modificar sensor}
  \item[Testes feitos:]
  \begin{enumerate}
    \item{Modificação do nome do sensor}
  \end{enumerate}
  \item[Problemas:]
  \begin{enumerate}
    \item{-}
  \end{enumerate}
  \item[Correções feitas:]
  \begin{enumerate}
    \item{-}
  \end{enumerate}  

  \section{Caso de teste 11: Excluir sensor}
  \item[Testes feitos:]
  \begin{enumerate}
    \item{Exclusão de um sensor}
  \end{enumerate}
  \item[Problemas:]
  \begin{enumerate}
    \item{-}
  \end{enumerate}
  \item[Correções feitas:]
  \begin{enumerate}
    \item{-}
  \end{enumerate}

  \section{Caso de teste 12: Listagem de sensores}
  \item[Testes feitos:]
  \begin{enumerate}
    \item{Visualização de um sensor}
  \end{enumerate}
  \item[Problemas:]
  \begin{enumerate}
    \item{-}
  \end{enumerate}
  \item[Correções feitas:]
  \begin{enumerate}
    \item{-}
  \end{enumerate}

  \section{Caso de teste 13: Criar meta}
  \item[Testes feitos:]
  \begin{enumerate}
    \item{Criação de meta sem medidas anteriores}
    \item{Criação de meta para meses que não são seguintes sem medidas anteriores}
    \item{Criação de meta para meses que não são seguintes com medidas anteriores (apenas o mês anterior)}
  \end{enumerate}
  \item[Problemas:]
  \begin{enumerate}
    \item{Nos testes 1 e 2 ocorreu um erro onde a meta não foi cadastrada e nenhum indicativo do erro aparecia}
    \item{No teste 3, pra criação da meta, foi considerado um mês errado como o de comparação}
  \end{enumerate}
  \item[Correções feitas:]
  \begin{enumerate}
    \item{Colocado uma mensagem de erro para tais ocasiões indicando qual o erro}
    \item{Foi restringida a criação de metas para o mes seguinte apenas}
  \end{enumerate}  

  \section{Caso de teste 14: Modificar meta}
  \item[Testes feitos:]
  \begin{enumerate}
    \item{Modificação do nome da meta}
    \item{Modificação da porcentagem relativa da meta com números acima de 100\%}
  \end{enumerate}
  \item[Problemas:]
  \begin{enumerate}
    \item{No teste 2, o valor foi usado normalmente para a meta}
  \end{enumerate}
  \item[Correções feitas:]
  \begin{enumerate}
    \item{O valor da meta foi concluído como um valor que é de responsabilidade do usuário, por isso foi retirado o limite desse de um máximo de 100}
  \end{enumerate}  

  \section{Caso de teste 15: Excluir meta}
  \item[Testes feitos:]
  \begin{enumerate}
    \item{Exclusão de uma meta}
  \end{enumerate}
  \item[Problemas:]
  \begin{enumerate}
    \item{-}
  \end{enumerate}
  \item[Correções feitas:]
  \begin{enumerate}
    \item{-}
  \end{enumerate}

  \section{Caso de teste 16: Listagem de metas}
  \item[Testes feitos:]
  \begin{enumerate}
    \item{Visualização de uma meta}
  \end{enumerate}
  \item[Problemas:]
  \begin{enumerate}
    \item{-}
  \end{enumerate}
  \item[Correções feitas:]
  \begin{enumerate}
    \item{-}
  \end{enumerate}

\end{description}