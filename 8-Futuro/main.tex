\chapter{Trabalhos Futuros}

O trabalho ainda não passa de um protótipo, tendo vários pontos de melhoria. Das melhorias possíveis do sistema, podem ser citadas:

\begin{itemize}
  \item{Controle da rede de sensores pela aplicação ou outros meios:} uma das funcionalidades envisionadas no início do projeto era a possibilidade de controlar a rede pela aplicação a rede para, por exemplo, controlar a periodicidade das amostragens dos módulos sensores ou forçar os módulos a dormir, o que possibilitaria ao sistema monitorar de uma maneira mais eficiente os equipamentos existentes. Tal funcionalidade é possivel devido a utilização de XBees no sistema.

  \item{Medição de potência real e reativa:} no trabalho, uma das premissas foi o uso de um fator de potência unitário, o que pode, na maioria dos casos, ser aproximada em residências. Porém, se tais grandezas fossem medidas seria possível um controle da qualidade da rede, aproximando o usuário à qualidade do serviço do distribuidor. Ou então o uso do sistema em empresas, que usam mais intensamente a rede.

  \item{Possibilitar vários usuários no sistema:} ao possibilitar a entrada de mais de um usuário no sistema possibilitaria escalar o protótipo para um produto onde vários usuários poderiam, por exemplo, comparar seus gastos com outras residências similares aumentando o grau de educação dele.
\end{itemize}