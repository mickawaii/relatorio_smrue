\chapter{Considerações Finais}
\label{Cap:consideracoes_finais}

O projeto permitiu o uso dos conhecimentos vistos em aula relacionados à microeletrônica, redes, sistemas digitais, engenharia de software, banco de dados, entre outros. Pelo fato da implementação ter envolvido uma grande variedade de conhecimentos cujas noções básicas foram ensinadas no percorrer da faculdade, o projeto foi muito interessante do ponto de vista de aprendizado, porém dificultou muito a implementação que divergiu um pouco da área de especialização do curso, mas ainda concentrando-se na engenharia.

O sistema tinha como objetivo prover ao usuário do sistema uma noção do consumo de cada equipamento para que esse tivesse mais controle sobre seus gastos. Além disso, o sistema deveria ser simples de montar e de baixo valor. Pode se dizer que tais objetivos foram alcançados, apesar do alto custo de investimento inicial (vide apêndice A), sendo que cada equipamento adicional a ser monitorado custaria um adicional de aproximadamente 200 reais (contando o custo do dólar em aproximadamente 4 reais). A teoria envolvida também pode ser considerada simples, dado que hoje em dia há uma diversidade de fontes abertas para o aprendizado. Portanto, o gargalo para a construção do sistema se torna mesmo a vontade que o usuário tem em aprender as técnicas, tecnologias e ferramentas envolvidas e a disponibilidade monetária que esse possui.

Em termos de dificuldades encontradas ao longo do trabalho pode ser citado a parte da implementação dos circuitos, uma vez que o projeto exigia o manuseio direto da rede elétrica, isso deixou os integrantes bem apreensivos, porém como foram tomados os devidos cuidados não foram encontrados problemas. A necessidade de importação de peças do exterior também apresentava um risco de atrasar o projeto, o que implicou em um planejamento mais cuidados dos integrantes quanto ao tempo.