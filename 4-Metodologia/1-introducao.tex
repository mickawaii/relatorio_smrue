\chapter{Metodologia}
\label{Métodos de implementação}

O trabalho foi desenvolvido com base de uma pesquisa de trabalhos acadêmicos, convergindo em um protótipo funcional. Para tanto, o trabalho foi dividido em algumas etapas principais:

\begin{itemize}
\item{Etapa 1: Pesquisa de trabalhos passados similares, agregação da base teórica envolvida;}
\item{Etapa 2: Reflexão sobre possíveis caminhos alternativos e desenvolvimento de planos paralelos redundantes;}
\item{Etapa 3: Implementação teórica da arquitetura do sistema, levantamento de peças necessárias, orçamento, preparação do ambiente de trabalho com ferramentas de depuração e testes;}
\item{Etapa 4: Implementação prática do planejamento feito na etapa anterior;}
\item{Etapa 5: Testes comparativos com os dados esperados;}
\item{Etapa 6: Finalização do projeto e do protótipo;}
\end{itemize}

Com essa metodologia, foi possível levantar os requisitos do sistema sem que houvessem mudanças bruscas no resultado final, tendo sempre um caminho paralelo a seguir caso ocorresse algum inesperado.